\documentclass{article}\usepackage[]{graphicx}\usepackage[]{color}
%% maxwidth is the original width if it is less than linewidth
%% otherwise use linewidth (to make sure the graphics do not exceed the margin)
\makeatletter
\def\maxwidth{ %
  \ifdim\Gin@nat@width>\linewidth
    \linewidth
  \else
    \Gin@nat@width
  \fi
}
\makeatother

\definecolor{fgcolor}{rgb}{0.345, 0.345, 0.345}
\newcommand{\hlnum}[1]{\textcolor[rgb]{0.686,0.059,0.569}{#1}}%
\newcommand{\hlstr}[1]{\textcolor[rgb]{0.192,0.494,0.8}{#1}}%
\newcommand{\hlcom}[1]{\textcolor[rgb]{0.678,0.584,0.686}{\textit{#1}}}%
\newcommand{\hlopt}[1]{\textcolor[rgb]{0,0,0}{#1}}%
\newcommand{\hlstd}[1]{\textcolor[rgb]{0.345,0.345,0.345}{#1}}%
\newcommand{\hlkwa}[1]{\textcolor[rgb]{0.161,0.373,0.58}{\textbf{#1}}}%
\newcommand{\hlkwb}[1]{\textcolor[rgb]{0.69,0.353,0.396}{#1}}%
\newcommand{\hlkwc}[1]{\textcolor[rgb]{0.333,0.667,0.333}{#1}}%
\newcommand{\hlkwd}[1]{\textcolor[rgb]{0.737,0.353,0.396}{\textbf{#1}}}%
\let\hlipl\hlkwb

\usepackage{framed}
\makeatletter
\newenvironment{kframe}{%
 \def\at@end@of@kframe{}%
 \ifinner\ifhmode%
  \def\at@end@of@kframe{\end{minipage}}%
  \begin{minipage}{\columnwidth}%
 \fi\fi%
 \def\FrameCommand##1{\hskip\@totalleftmargin \hskip-\fboxsep
 \colorbox{shadecolor}{##1}\hskip-\fboxsep
     % There is no \\@totalrightmargin, so:
     \hskip-\linewidth \hskip-\@totalleftmargin \hskip\columnwidth}%
 \MakeFramed {\advance\hsize-\width
   \@totalleftmargin\z@ \linewidth\hsize
   \@setminipage}}%
 {\par\unskip\endMakeFramed%
 \at@end@of@kframe}
\makeatother

\definecolor{shadecolor}{rgb}{.97, .97, .97}
\definecolor{messagecolor}{rgb}{0, 0, 0}
\definecolor{warningcolor}{rgb}{1, 0, 1}
\definecolor{errorcolor}{rgb}{1, 0, 0}
\newenvironment{knitrout}{}{} % an empty environment to be redefined in TeX

\usepackage{alltt}
\usepackage{Sweave}
\usepackage{float}
\usepackage{graphicx}
\usepackage{tabularx}
\usepackage{siunitx}
\usepackage{mdframed}
\usepackage{natbib}
\bibliographystyle{..//refs/styles/besjournals.bst}
\usepackage[small]{caption}
\setkeys{Gin}{width=0.8\textwidth}
\setlength{\captionmargin}{30pt}
\setlength{\abovecaptionskip}{0pt}
\setlength{\belowcaptionskip}{10pt}
\topmargin -1.5cm        
\oddsidemargin -0.04cm   
\evensidemargin -0.04cm
\textwidth 16.59cm
\textheight 21.94cm 
%\pagestyle{empty} %comment if want page numbers
\parskip 7.2pt
\renewcommand{\baselinestretch}{1.5}
\parindent 0pt

\newmdenv[
  topline=true,
  bottomline=true,
  skipabove=\topsep,
  skipbelow=\topsep
]{siderules}

%% R Script


\IfFileExists{upquote.sty}{\usepackage{upquote}}{}
\begin{document}


\renewcommand{\thetable}{\arabic{table}}
\renewcommand{\thefigure}{\arabic{figure}}
\renewcommand{\labelitemi}{$-$}

%%%%%%%%%%%%%%%%%%%%%%%%%%%%%%%%%%%%%%%%%%%%%%%
\section*{Understanding (Defining?) Vegetative Risk}
Lizzie: ``Phenophases, Species differences (maybe just set up here...), regional differences?, some of your points in Box 1 here - basically set up everything that could matter then make a case for what matters most and hit on those in next sections.''

Another highly crucial factor to consider is the rate of budburst and the length of time between budburst to full leafout, which we will refer to as the duration of vegetative risk. % repeated later in the subsection: Species Differences and Vegetative Risk, can probably take it out here

\subsection*{Phenophases}
The level of damage sustained by plants from a false spring also varies across phenophases. Generally, reproductive phases are more sensitive to false spring events than vegetative phases and developing leaves are more susceptible to damage than opening buds or expanding shoots \citep{Lenz2013,Augspurger2009}. However, trees that suffer severe vegetative growth damage from a false spring event will suffer greater long-term effects from the loss of photosynthetic tissue than trees that lose one year of reproductive growth. Spring frosts during the vegetative growth phenophases impose the greatest freezing threat to deciduous tree species \citep{Sakai1987}.

Phenophase is a greater indicator for level of risk than life stage. Individuals at a certain phenophase (i.e. between budburst and full leafout) are more likely to incur damage from a freezing event than individuals past the leafout phenophase, independent of life stage \citep{Augspurger2009,Vitasse2014}.

\subsection*{Species}
Seedlings and saplings initiate budburst before canopy closure in order to benefit from the increased light levels \citep{Augspurger2008}, therefore putting them at greater risk to false spring damage than adult trees \citep{Vitasse2014}. Younger trees are more likely to incur lastly damage to the leaf buds and vegetative growth, whereas adult trees are at risk of xylem embolism. In order for xylem embolism to occur, extreme cavitation must first occur. Extensive cavitation in the xylem would require more intensive freezing events than it would take to damage seedling and sapling leaf buds. Especially strong freezing events (i.e. >-8.6$^{\circ}$C), could result in meristemic tissue, wood parenchyma and phloem damage \citep{Lenz2013, Augspurger2011, Sakai1987}.  

However, different species respond differently to anthropogenic climate change. Most species are expected to begin leaf out earlier in the season with warming spring temperatures but some species may have the opposite response \citep{Xin2016, Cleland2006, Yu2010}.

Studies indicate that species growing at more northern latitudes tend to respond greater to photoperiod than species growing further south \citep{Caffarra2011, Viheraaarnio2006, Partanen2004}. Similarly, late successional species exhibit greater photoperiod sensitivities than pioneer species \citep{Basler2012} and they also require more chilling in the winter and greater forcing temperatures in the spring \citep{Laube2013}. 

\section*{Species Differences and Vegetative Risk}
Lizzie: ``You have so much great stuff here but need to develop WHY and HOW species would differ. You touch on this once in paper but need to develop more. This section should start with building a case for why species would differ then go into data.''

Plants are most susceptible to frost damage between budburst and leafout \citep{Lenz2016, Vitasse2014, Augspurger2009}. The rate of budburst and the length of time between budburst and leafout is a crucial indicator for predicting level of damage from a false spring event. We will refer to the timing of these collective phenophases (i.e. budburst to leafout) as the duration of vegetative risk. The duration of vegetative risk generally is extended if a freezing event occurs during the phenophases between budburst and leaf expansion and species with short durations of vegetative risk often sustain higher levels of damage. Therefore, if the duration of vegetative risk is longer, then the buds and leaves will be heartier against frosts, however this still has yet to be tested thoroughly \citep{Augspurger2009}. Frost tolerance, however, steadily decreases after budburst begins until the leaf is fully unfolded, with leafout being the most susceptible to frost damage \citep{Lenz2016}. It is therefore crucial that more studies investigate the relationship between false spring events and duration of vegetative risk. 

It is anticipated that these more opportunistic individuals that initiate budburst earlier in the spring with anthropogenic climate change would attempt to limit freezing risk by decreasing their duration of vegetative risk and progress to full leaf expansion faster than larger or more northern species. We assess this interaction using three datasets: from a citizen science program, a growth chamber chilling experiment, and long-term observational data. 

\subsection*{Treespotters Data}
We analyzed a dataset from a USA-NPN citizen science program, the Arnold Arboretum Tree Spotters %Should I mention here that it is our program or no? Also, not exactly sure how to cite it...
, to discern the relationship between duration of vegetative risk and initial budburst date. Figure 2 shows the duration of vegetative risk for 11 different species observed at the Arnold Arboretum in 2016. \textit{Quercus alba} and \textit{Betula nigra} had the longest durations of vegetative risk, most likely due to the fact that one is late successional species and the other is northern species respectively. Overall, there is no significant relationship between duration of vegetative risk and day of budburst. This could be due to the fact that there are various functional groups involved in this study, that it is just over the course of one year, that it is in an arboretum, and that it is a citizen science project. Further investigations should be made to gain a better understanding of observational studies for this interaction. 

\subsection*{Dan's Data}
It is possible, that with anthropogenic climate change progressing, leaf out timing may be delayed. As winter seasons begin to warm and chilling requirements are not met, more warming in the spring must first occur for budburst to begin \citep{Chuine2010, Polgar2014, Fu2012, Morin2009, McCreary1990}. In a chilling experiment , there were various experimental chilling, photoperiod, and forcing treatments (Flynn \& Wolkovich, ?). In Figure 2, five species were assessed across five different treatments. C is a forcing temperature of 15$^{\circ}$C during the day and 5$^{\circ}$C at night, W is a forcing temperature of 20$^{\circ}$C during the day and 10$^{\circ}$C at night, S is a short day with 8 hours of daylight, L is a long day with 12 hours of daylight, 0 is no additional winter chilling, 1 is 33 days of additional winter chilling at 4$^{\circ}$C, and 2 is 33 days of additional winter chilling at 1.5$^{\circ}$C. QUERUB is \textit{Quercus rubra}, ACERUB is \textit{Acer rubrum}, POPGRA is \textit{Populus grandidentata}, ILEMUC is \textit{Ilex mucronata}, BETPAP is \textit{Betula papyrifera}. Anova results indicate that chilling has a similar effect on budburst and leafout but photoperiod and forcing temperatures have varying effects on the two phenophases resulting in changes in duration of vegetative risk. This should be analyzed further in future studies. 

As seen in Table 5, the warmer forcing temperatures (20$^{\circ}$C during the day and 5$^{\circ}$ at night) had the greatest affect on duration of vegetative risk. High forcing temperatures greatly reduced the length of time it took between budburst and leaf out for all species. 

\subsection*{Harvard Forest Data}
The final dataset for measuring duration of vegetative risk against initial day of budburst is from John O'Keefe's observational data that was also used in the \textit{Determining Spring Onset} section. For this portion of the study, we assessed two years of data: one year that had an unusually early spring onset (2010) and another year that had an unusually late spring (2014). QUAL is \textit{Quercus alba}, FRAM is \textit{Fraxinus americana}, BEAL is \textit{Betula alleghaniensis}, ACRU is \textit{Acer rubrum}, FAGR is \textit{Fagus grandifolia}, ACPE is \textit{Acer pensylvanicum}, QURU is \textit{Quercus rubra}, and HAVI is \textit{Hamamelis virginiana}. As is evident in Figure 4, the duration of vegetative risk is slightly longer in 2010 which was when spring onset was unusually early, which could be from lower forcing temperatures. Given the paucity of information and the disparate results across the three studies assessing duration of vegetative risk, more research should be done in order to better understand this relationship. 

\bibliography{..//refs/SpringFreeze.bib}
\end{document}
