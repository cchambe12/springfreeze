\documentclass{article}\usepackage[]{graphicx}\usepackage[]{color}
%% maxwidth is the original width if it is less than linewidth
%% otherwise use linewidth (to make sure the graphics do not exceed the margin)
\makeatletter
\def\maxwidth{ %
  \ifdim\Gin@nat@width>\linewidth
    \linewidth
  \else
    \Gin@nat@width
  \fi
}
\makeatother

\definecolor{fgcolor}{rgb}{0.345, 0.345, 0.345}
\newcommand{\hlnum}[1]{\textcolor[rgb]{0.686,0.059,0.569}{#1}}%
\newcommand{\hlstr}[1]{\textcolor[rgb]{0.192,0.494,0.8}{#1}}%
\newcommand{\hlcom}[1]{\textcolor[rgb]{0.678,0.584,0.686}{\textit{#1}}}%
\newcommand{\hlopt}[1]{\textcolor[rgb]{0,0,0}{#1}}%
\newcommand{\hlstd}[1]{\textcolor[rgb]{0.345,0.345,0.345}{#1}}%
\newcommand{\hlkwa}[1]{\textcolor[rgb]{0.161,0.373,0.58}{\textbf{#1}}}%
\newcommand{\hlkwb}[1]{\textcolor[rgb]{0.69,0.353,0.396}{#1}}%
\newcommand{\hlkwc}[1]{\textcolor[rgb]{0.333,0.667,0.333}{#1}}%
\newcommand{\hlkwd}[1]{\textcolor[rgb]{0.737,0.353,0.396}{\textbf{#1}}}%
\let\hlipl\hlkwb

\usepackage{framed}
\makeatletter
\newenvironment{kframe}{%
 \def\at@end@of@kframe{}%
 \ifinner\ifhmode%
  \def\at@end@of@kframe{\end{minipage}}%
  \begin{minipage}{\columnwidth}%
 \fi\fi%
 \def\FrameCommand##1{\hskip\@totalleftmargin \hskip-\fboxsep
 \colorbox{shadecolor}{##1}\hskip-\fboxsep
     % There is no \\@totalrightmargin, so:
     \hskip-\linewidth \hskip-\@totalleftmargin \hskip\columnwidth}%
 \MakeFramed {\advance\hsize-\width
   \@totalleftmargin\z@ \linewidth\hsize
   \@setminipage}}%
 {\par\unskip\endMakeFramed%
 \at@end@of@kframe}
\makeatother

\definecolor{shadecolor}{rgb}{.97, .97, .97}
\definecolor{messagecolor}{rgb}{0, 0, 0}
\definecolor{warningcolor}{rgb}{1, 0, 1}
\definecolor{errorcolor}{rgb}{1, 0, 0}
\newenvironment{knitrout}{}{} % an empty environment to be redefined in TeX

\usepackage{alltt}
\usepackage{Sweave}
\usepackage{float}
\usepackage{graphicx}
\usepackage{tabularx}
\usepackage{siunitx}
\usepackage{mdframed}
\usepackage{natbib}
\bibliographystyle{..//refs/styles/besjournals.bst}
\usepackage[small]{caption}
\setkeys{Gin}{width=0.8\textwidth}
\setlength{\captionmargin}{30pt}
\setlength{\abovecaptionskip}{0pt}
\setlength{\belowcaptionskip}{10pt}
\topmargin -1.5cm        
\oddsidemargin -0.04cm   
\evensidemargin -0.04cm
\textwidth 16.59cm
\textheight 21.94cm 
%\pagestyle{empty} %comment if want page numbers
\parskip 7.2pt
\renewcommand{\baselinestretch}{1.5}
\parindent 0pt

\newmdenv[
  topline=true,
  bottomline=true,
  skipabove=\topsep,
  skipbelow=\topsep
]{siderules}

%% R Script


\IfFileExists{upquote.sty}{\usepackage{upquote}}{}
\begin{document}
\title{Rethinking False Spring Risk}
\author{C. J. Chamberlain $^{1,2}$, E. M. Wolkovich $^{1,2}$}
\date{\today}
\maketitle 
 

\renewcommand{\thetable}{\arabic{table}}
\renewcommand{\thefigure}{\arabic{figure}}
\renewcommand{\labelitemi}{$-$}

%%%%%%%%%%%%%%%%%%%%%%%%%%%%%%%%%%%%%%%%%%%%%%%
\section*{Introduction}
\begin{enumerate}
\item Introduce False Spring Concept
\begin {enumerate}
\item Plants growing in temperate environments are at risk of being exposed to late spring freezes, which can be detrimental to growth. 
\item Individuals that leaf out before the last frost are at risk of leaf loss, damaging wood tissue, and slowed or stalled canopy development \citep{Gu2008, Hufkens2012}. 
\item Therefore, temperate deciduous tree species must have plastic phenological responses in the spring in order to optimize photosynthesis and minimize frost or drought risk \citep{Polgar2011}. 
\item These late spring freezing events are known as false springs. False spring events can result in highly adverse ecological and economic consequences \citep{Knudson2012, Ault2013}.
\end{enumerate}
\item Introduce Climate Change and Importance of False Spring Studies
\begin{enumerate}
\item Climate change is expected to increase damage from false spring events around the world due to earlier spring onset and greater fluctuations in temperature \citep{Cannell1986, Inouye2008, Martin2010}. 
\item Temperate forest species around the world are initiating leaf out about 4.6 days earlier per degree Celsius \citep{Wolkovich2012, Polgar2014}. 
\item It is anticipated that there will be a decrease in false spring frequency overall but the magnitude of temperature variation is likely to increase, therefore amplifying the expected intensity of false spring events \citep{Kodra2011, Allstadt2015}. 
\item Multiple studies have documented false spring events in recent years \citep{Gu2008, Augspurger2009, Knudson2012, Augspurger2013} and some have linked this to climate change \citep{Ault2013, Allstadt2015, Muffler2016, Xin2016}. 
\item Due to these reasons, it is crucial for researchers to properly evaluate the effects of false spring events on temperate forests and agricultural crops in order to make more accurate predictions on future trends.
\end{enumerate}

\item Introduce Current False Spring Index Equation
\begin{enumerate}
\item Different species respond differently to late spring freezing events. 
\item The level of damage sustained by plants from a false spring also varies across phenophases. 
\item Various studies have assessed the risk of damage or the intensity of particular false spring events but at this time false spring studies fail to incorporate all potential factors that could affect the level of frost damage risk. 
\item A False Spring Index (FSI) signifies the likelihood of a damage to occur from a late spring freeze. 
\item Currently, FSI evaluates day of budburst, number of growing degree days, and day of last spring freeze through a simple equation as seen below \citep{Marino2011}. 
\begin{equation} \label{eq:1}
FSI = Julian Date (Last Spring Freeze) - Julian Date (Budburst)
\end{equation}
\item False spring studies largely simplify the various ecological elements that could predict the level of plant damage from late spring freezing events. 
\item In contrast to these simplifications, we argue that a wealth of factors greatly impacts plants' frost spring risk such that simple indices will most likely lead to inaccurate predictions and ultimately do little to advance the field. 
\end{enumerate}

\item State the Purpose of the Paper
\begin{enumerate}
\item In this paper we aim to highlight the complexity of factors driving a plant's false spring risk. 
\item We outline in particular how life stage of the individual \citep{Caffarra2011}, location within a forest or canopy \citep{Augspurger2013}, winter chilling hours (Flynn \& Wolkovich 2017?), proximity to water \citep{Gu2008}, level of precipitation prior to the freezing event \citep{Anderegg2013}, freeze duration/intensity, and range limits of the species \citep{Martin2010} unhinge simple metrics of false spring. 
\item The ultimate intent is to demonstrate how an integrated view of false spring that incorporates these factors would rapidly advance progress in this field.  
\end{enumerate}
\end{enumerate}

%%%%%%%%%% DEFINING FALSE SPRING %%%%%%%%%%%%%
\section*{Defining False Spring}
\begin{enumerate}
\item Definition and Threat
\begin{enumerate}
\item Episodic spring frosts present the most risk to frost damage for temperate forest plants \citep{Sakai1987}.
\item Abnormally warm conditions in the late winter or early spring can cause budburst to initiate early in trees and shrubs.
\item Freezing temperatures following a warm spell could result in plant damage or even death \citep{Ludlum1968, Mock2007}.
\item False springs are defined by two phases: rapid vegetative growth prior to a freeze and a post freeze setback \citep{Gu2008}.
\item Freeze and thaw fluctuations can cause defoliation, xylem embolism and decreased xylem conductivity which can result in crown dieback \citep{Gu2008}.
\item Species that are better able to phenologically track the shifts in spring advancement due to climate change are more likely to sustain damaging events such as false springs \citep{Scheifinger2003}.
\end {enumerate}

\item Define Chilling requirements to specify timing of damaging false spring events
\begin {enumerate}
\item Deciduousness and the evolution of two dormancy phases (i.e. endodormancy and ecodormancy) in temperate forest trees has permitted species to occupy more northern ecological niches \citep{Samish1954}.
\item Endodormancy is the period of winter when temperate trees are inhibited from growing, regardless of the outdoor environment.
\item Ecodormancy is the period of time when growth can occur but the external environment is not conducive to growth (e.g. too cold) \citep{Basler2012}.
\item Therefore, warm temperatures earlier in the year (i.e. in February) do not seem to affect species, most likely because trees have not yet left the endodormancy phase.
\item Frost damage usually occurs when there is a warmer than average March, a freezing April, and enough growing degree days between budburst and the last freeze date \citep{Augspurger2013}.
\item A damaging false spring is currently defined as having more than 7 days between budburst and the last freeze date, using Equation 1 \citep{Peterson2014}.
\item The 7 day parameter exposes less resistant foliate phenophases to a false spring, thus putting the plant at a higher risk of damage. 
\item Once budburst has initiated, buds cannot respond to cold temperatures and freeze resistance is greatly reduced \citep{Taschler2004, Lenz2013, Vitasse2014}.
\item There are two types of freezes: a ``hard freeze" at -2.2$^{\circ}$C and a ``soft freeze" at -1.7$^{\circ}$C \citep{Vavrus2006, Kodra2011, Augspurger2013}.
\item However, the definition is still largely under debate. 
\end{enumerate}

\item Damage and drought
\begin{enumerate}
\item Freezing damage can occur directly via intracellular ice formation or indirectly via freezing dehydration \citep{Pearce2001, Beck2004, Hofmann2015}.
\item Intracellular ice formation often results in defoliaiton and increased xylem cavitation or embolism in the stem.
\item Freezing tolerance in plants is usually against extracellular freezing or freezing dehydration \citep{Burke1976}.
\item Drought and desiccation within the xylem mimick the adverse effects of false spring events \citep{Cavender2015}.
\item Dry winters typically result in new, frost-tolerant shoots due to the decreased water content and osmotic potential from the reduced number of accumulated solutes \citep{Morin2007, Hofmann2015}.
\item Therefore, it is hypothesized that increased bud dehydration results in increased frost hardiness \citep{Beck2007, Nielsen2009, Poirier2010, Kathke2011, Hofmann2015}.
\item However, more studies are needed to investigate the interplay between false spring events and precipiation and how that relationship impacts the level of damage a plant sustains. 
\end{enumerate}
\end{enumerate}




\bibliography{..//refs/SpringFreeze.bib}
\end{document}
