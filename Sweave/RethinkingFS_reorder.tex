\documentclass{article}\usepackage[]{graphicx}\usepackage[]{color}
%% maxwidth is the original width if it is less than linewidth
%% otherwise use linewidth (to make sure the graphics do not exceed the margin)
\makeatletter
\def\maxwidth{ %
  \ifdim\Gin@nat@width>\linewidth
    \linewidth
  \else
    \Gin@nat@width
  \fi
}
\makeatother

\definecolor{fgcolor}{rgb}{0.345, 0.345, 0.345}
\newcommand{\hlnum}[1]{\textcolor[rgb]{0.686,0.059,0.569}{#1}}%
\newcommand{\hlstr}[1]{\textcolor[rgb]{0.192,0.494,0.8}{#1}}%
\newcommand{\hlcom}[1]{\textcolor[rgb]{0.678,0.584,0.686}{\textit{#1}}}%
\newcommand{\hlopt}[1]{\textcolor[rgb]{0,0,0}{#1}}%
\newcommand{\hlstd}[1]{\textcolor[rgb]{0.345,0.345,0.345}{#1}}%
\newcommand{\hlkwa}[1]{\textcolor[rgb]{0.161,0.373,0.58}{\textbf{#1}}}%
\newcommand{\hlkwb}[1]{\textcolor[rgb]{0.69,0.353,0.396}{#1}}%
\newcommand{\hlkwc}[1]{\textcolor[rgb]{0.333,0.667,0.333}{#1}}%
\newcommand{\hlkwd}[1]{\textcolor[rgb]{0.737,0.353,0.396}{\textbf{#1}}}%
\let\hlipl\hlkwb

\usepackage{framed}
\makeatletter
\newenvironment{kframe}{%
 \def\at@end@of@kframe{}%
 \ifinner\ifhmode%
  \def\at@end@of@kframe{\end{minipage}}%
  \begin{minipage}{\columnwidth}%
 \fi\fi%
 \def\FrameCommand##1{\hskip\@totalleftmargin \hskip-\fboxsep
 \colorbox{shadecolor}{##1}\hskip-\fboxsep
     % There is no \\@totalrightmargin, so:
     \hskip-\linewidth \hskip-\@totalleftmargin \hskip\columnwidth}%
 \MakeFramed {\advance\hsize-\width
   \@totalleftmargin\z@ \linewidth\hsize
   \@setminipage}}%
 {\par\unskip\endMakeFramed%
 \at@end@of@kframe}
\makeatother

\definecolor{shadecolor}{rgb}{.97, .97, .97}
\definecolor{messagecolor}{rgb}{0, 0, 0}
\definecolor{warningcolor}{rgb}{1, 0, 1}
\definecolor{errorcolor}{rgb}{1, 0, 0}
\newenvironment{knitrout}{}{} % an empty environment to be redefined in TeX

\usepackage{alltt}
\usepackage{Sweave}
\usepackage{float}
\usepackage{graphicx}
\usepackage{tabularx}
\usepackage{siunitx}
\usepackage{geometry}
\usepackage{pdflscape}
\usepackage{mdframed}
\usepackage{natbib}
\bibliographystyle{..//refs/styles/besjournals.bst}
\usepackage[small]{caption}
\setlength{\captionmargin}{30pt}
\setlength{\abovecaptionskip}{0pt}
\setlength{\belowcaptionskip}{10pt}
\topmargin -1.5cm        
\oddsidemargin -0.04cm   
\evensidemargin -0.04cm
\textwidth 16.59cm
\textheight 21.94cm 
%\pagestyle{empty} %comment if want page numbers
\parskip 7.2pt
\renewcommand{\baselinestretch}{1.5}
\parindent 0pt
\usepackage{lineno}
\linenumbers

\newmdenv[
  topline=true,
  bottomline=true,
  skipabove=\topsep,
  skipbelow=\topsep
]{siderules}

%% R Script


\IfFileExists{upquote.sty}{\usepackage{upquote}}{}
\begin{document}
\title{Rethinking False Spring Risk}
\author{C. J. Chamberlain $^{1,2}$, B. I. Cook $^{3}$, I. Garcia de Cortazar Atauri $^{4}$ \& E. M. Wolkovich $^{1,2}$}
\date{\today}
\maketitle 
%\tableofcontents
 

\renewcommand{\thetable}{\arabic{table}}
\renewcommand{\thefigure}{\arabic{figure}}
\renewcommand{\labelitemi}{$-$}
\setkeys{Gin}{width=0.8\textwidth}

%%%%%%%%%%%%%%%%%%%%%%%%%%%%%%%%%%%%%%%%%%%%%%%
% General to do
% Move all figures and their captions to end of manuscript
% Work on transitions throughout. I made note of it many places.
% My comments are usually in [] and I made some edits throughout. You can use the app FileMerge (spotlight search for it) on most Macs to see the changes quickly. 
%%%%%%%%%%%%%%%%%%%%%%%%%%%%%%%%%%%%%%%%%%%%%%%


\section{Abstract}
Trees and shrubs growing in temperate environments are at risk of being exposed to late spring freezes---often called false spring events---which can be damaging ecologically and economically. As climate change may alter the potential prevalence and severity of false spring events, our ability to accurately forecast such events has become more critical. Yet currently, many false spring studies simplify the various ecological elements needed for accurate predictions of the level of plant damage from late spring freezing events. Here we review the complexity of factors driving a plant's false spring risk. In particular we highlight how integrating species, life stage, and habitat differences could help accurately determine the level of damage sustained by a false spring event. %(The ultimate intent is to demonstrate how an integrated view of false spring that incorporates these factors would rapidly advance progress in this field.)
%EMW: You mention equation above, but have not told the audience about any equations, re-work the sentence/paragraph (probably best to remove 'equation').
Integrating some of these complexities could help rapidly advance understanding and forecasting of false spring events in climate change and ecological studies.
% In this paper we aim to highlight the complexity of factors driving a plant's false spring risk and provide a road map for improved metrics. First, we review the currently used definitions of false spring. Then, combining research from plant physiology, climatology and community ecology, we outline major gaps in current definitions.

\section{Introduction}

Plants growing in temperate environments time their growth each spring to follow rising temperatures and increasing light and soil resource availability. While tracking spring resource availability, temperate plants are at risk of late spring freezes, which can be detrimental to growth. Individuals that leaf out before the last freeze date are at risk of leaf loss, damaged wood tissue, and slowed or stalled canopy development \citep{Gu2008, Hufkens2012}. These damaging late spring freezing events are known as false springs, and are widely documented to result in highly adverse ecological and economic consequences \citep{Knudson2012, Ault2013}.

% EMW: Thanks for adding references to the temperature variation. Unfortunately I am not comfortable with these references, especially the Vasseur one (remind me and we can discuss briefly). I suggest you remove this one and add 1-2 from the climate literature. My understanding is temperature variation is not going to change that much in most regions, and I don't want to support a narrative that it will when it won't. An easy fix for now is deleting a couple sentences I am not sure we need (deleted below, see what you think). But in the longer run I suggest you Skype with Ben to discuss more and track down some balanced citations from the physical science literature.
Climate change is expected to cause an increase in damage from false spring events due to earlier spring onset and potentially greater fluctuations in temperature in some regions \citep{Cannell1986, Inouye2008, Martin2010}. Already, multiple studies have documented false spring events in recent years \citep{Gu2008, Augspurger2009, Knudson2012, Augspurger2013} and some have linked these events to climate change \citep{Ault2013, Allstadt2015, Muffler2016, Xin2016}. This increasing interest in false spring events has led to a growing body of research investigating the effects on temperate forests and agricultural crops. But for this research to produce accurate predictions of future trends, researchers need methods that properly evaluate the effects of false spring events across the diverse species, habitats and climate regimes they are studying. 

Current metrics for estimating false springs events are generally simple: often requiring an estimate for the start of `spring' and whether temperatures occurred below a particular temperature threshold in the following week. Such estimates inherently assume consistency of responses across species, functional group, life stages, habitat type, and other climatic regimes, ignoring that such factors can greatly impact plants' false spring risk. As a result, such indices will most likely lead to inaccurate current estimates as well as poor future predictions, slowing our progress in understanding false spring events and how they may shift with climate change. 

In this paper we aim to highlight the complexity of factors driving a plant's false spring risk and provide a road map for improved metrics. First, we review the currently used definitions of false spring. Then, combining research from plant physiology, climatology and community ecology, we outline major gaps in current definitions. In particular we show how life stage \citep{Caffarra2011}, location within a forest or canopy \citep{Augspurger2013}, interspecific variation in avoidance and tolerance strategies (Flynn \& Wolkovich 2017?), freeze temperature thresholds, and regional effects \citep{Martin2010} unhinge simple metrics of false spring. We argue that a new approach that integrates these and other crucial factors would help accurately determine current false spring damage and improve predictions of spring freeze risk under a changing climate---while potentially providing novel insights to how plants respond to and are shaped by spring frost. % The ultimate intent is to demonstrate how an integrated view of false spring that incorporates these factors would rapidly advance progress in this field.  

\section{Defining False Spring: An example in one temperate plant community}
Temperate forest plants experience elevated risk of frost damage during the spring due to the stochastic timing of spring frosts. 
% Temperate forest plants are most at risk to frost damage from episodic spring frosts \citep{Sakai1987}. 
Plants thus must exhibit flexible spring phenologies in order to minimize freezing risk. Freezing temperatures following a warm spell could result in plant damage or even death \citep{Ludlum1968, Mock2007}. Intracellular ice formation from false spring events often results in severe leaf and stem damage. Ice formation can also occur indirectly (i.e. extracellularly), which results in freezing dehydration and mimics extreme drought conditions \citep{Pearce2001, Beck2004, Hofmann2015}. Both forms of ice formation can cause defoliation and, ultimately, crown dieback \citep{Gu2008}. Once buds exit the dormancy phase, they are less freeze tolerant and resistance to bud ice formation is greatly reduced \citep{Taschler2004, Lenz2013, Vitasse2014a}. An effective and consistent definition of false spring would accurately determine the amount and type of ice formation to properly evaluate the level of damage that could occur.

% EMW: Tiny edits below...
There are several definitions currently used to define a false spring. A common definition describes a false spring as having two phases: rapid vegetative growth prior to a freeze and a post freeze setback \citep{Gu2008}. Other definitions instill more precise temporal parameters, specific to certain regions \citep[e.g., in][false spring for the Midwestern United States is defined as a warmer than average March, a freezing April, and enough growing degree days between budburst and the last freeze date]{Augspurger2013}. A widely used definition integrates a mathematical equation to quantify a false spring event. This equation, known as a False Spring Index (FSI), signifies the likelihood of damage to occur from a late spring freeze. Currently, FSI is evaluated by the day of budburst and the day of last spring freeze through the simple equation \citep{Marino2011}:
\begin{equation} \label{eq:1}
FSI = \text{Day of Year} (Last Spring Freeze) - \text{Day of Year} (Budburst)
\end{equation}
A damaging FSI is currently defined to be 7 or more days between budburst and the last freeze date (Equation \ref{eq:1}) \citep{Peterson2014}. By using the 7 day threshold, it is likely less resistant vegetative phenophases will be exposed to a false spring, thus putting the plant at a higher risk of damage. %FIX: Threshold does not expose, need a few more words.

To demonstrate how the FSI definition works, we applied it to data from the Harvard Forest Long-term Ecological Research program in Massachusetts. We used three separate methodologies to calculate spring onset: long-term ground observational data \citep{Okeefe2014}, PhenoCam data from Harvard Forest \citep{Richardson2015}, and USA National Phenology Network (USA-NPN) Extended Spring Index (SI-x) data \citep{USA-NPN2016}. These spring onset values were then inputted into the FSI equation (Equation \ref{eq:1}) to calculate FSI from 2008 to 2014 (Figure \ref{fig:fsifig}). 

% EMW: Tiny edits below...
Each methodology renders different FSI values, suggesting different false spring damage for the same site and same year. For most years, the observational FSI and PhenoCam FSI are about 10-15 days lower than the SI-x data. This is especially important for 2008, when the SI-x data indicates a false spring year, whereas the other two datasets do not. In 2012, the observational data and PhenoCam data diverge and the PhenoCam FSI is over 30 days less than the SI-x value.

The reason for these discrepancies is that each method evaluates spring onset for different species or functional groups within a forest community. Spring phenology in temperate forests typically progresses by functional group: understory species and young trees tend to initiate budburst first, whereas larger canopy species may start later in the season \citep{Richardson2009, Xin2016}. The different FSI values determined in Figure \ref{fig:fsifig} exemplify the differences in functional group spring onset dates and illustrate variations in forest demography and phenology, which is most apparent in 2012. In 2012, a false spring event was reported through many regions of the US due to warm temperatures occurring in March \citep{Ault2015}. These high temperatures would most likely be too early for larger canopy species to initiate budburst but they would affect smaller understory species as is seen in Figure 1. 

% EMW: Tiny edits below...
Yet, in contrast to our three metrics of spring onset for one site, most FSI work currently ignores variation across functional groups --- instead using one metric of spring onset and assuming it applies to the whole community of plants. The risk of a false spring varies across habitats and with species composition since spring onset is not consistent across functional groups. Therefore, one spring onset date cannot be used as an effective proxy for all species. False spring studies should first assess the forest demographics and functional groups relevant to the study question in order to effectively estimate the date of spring onset. However, as we outline below, considering different functional groups is unlikely to be enough for robust predictions. It is also crucial to integrate species differences within functional groups and consider the various interspecific avoidance and tolerance strategies that species have against false springs. %EMW cut after re-order: ... equation is built half on spring onset date, it is critical to choose the most appropriate methodology that targets the species of interest within the study. 

\section {Plant Physiology and Diversity versus the Current False Spring Definition}
% EMW: Tiny edits below...
Plants have evolved to minimize false spring damage through two strategies: avoidance and tolerance. Effective avoidance strategies require well-timed spring phenologies. Temperate deciduous tree species optimize growth and minimize spring freeze damage by using three cues to initiate budburst: low winter temperatures, warm spring temperatures, and increasing photoperiods \citep{Chuine2010}. The evolution of these three cues and their interactions has permitted temperate plant species to occupy more northern ecological niches and decrease the risk of false spring damage, which is crucial for avoidance strategies \citep{Samish1954}. One avoidance strategy, for example, is the interaction between over-winter chilling and spring forcing temperatures. Warm temperatures earlier in the year (i.e. in February) will not result in early budburst due to insufficient chilling \citep{Basler2012}. Likewise, photoperiod sensitivity is a common false spring avoidance strategy: species that respond strongly to photoperiod cues in addition to warm spring temperatures will likely delay budburst and evade false spring events as spring continues to advance earlier in the year \citep{Basler2014}. 

Some temperate forest species have evolved to be more tolerant of spring freezing temperatures, rather than try to avoid frosts via more flexible phenologies. Temperate forest plants utilize various morphological strategies to be more frost tolerant: some have toothed or lobed leaves to increase `packability' in winter buds, which permits more rapid leafout and minimizes exposure time of less resistant tissues \citep{Edwards2017}. Other species have young leaves with more trichomes to act as a buffer against spring frosts \citep{Agrawal2004, Prozherina2003}, and many are able to respond to abiotic cues such as consistently dry winters. Species living in habitats with drier winters develop shoots and buds with decreased water content, which makes the buds more tolerant to drought and also to false spring events \citep{Beck2007, Morin2007, Nielsen2009, Poirier2010, Kathke2011, Hofmann2015}. More studies are needed to investigate the interplay between false spring events, leaf morphology, and precipitation and how these relationships affect false spring tolerance. Given the diverse array of spring freezing defense mechanisms, predicting damage by false spring events requires a greater understanding of avoidance and tolerance strategies across species, especially with a changing climate.

\section {Defining Vegetative Risk: Complexities due to Species' Strategies and Climate}
Phenology and frost tolerance are intertwined---with important variation occurring across different phenological phases. Different phenophases respond differently to false spring events, with flowering and fruiting being generally more sensitive than vegetative phases \citep{Augspurger2009, Lenz2013}.
However, false spring events that occur during the vegetative growth phenophases impose the greatest freezing threat to deciduous tree and shrub species. Plants will suffer greater long-term effects from the loss of photosynthetic tissue, which could impact multiple years of growth, reproduction, and canopy development \citep{Sakai1987, Vitasse2014}. 

There is also important variation within certain phenological phases. Most notably, within the vegetative phases of spring leafout, plants that have initiated budburst but have not fully leafed out are more likely to sustain damage from a false spring than individuals past the leafout phase. This is because freezing tolerance steadily decreases after budburst begins until the leaf is fully unfolded \citep{Lenz2016} (Figure \ref{fig:risk}). Therefore, the rate of budburst and the length of time between budburst and leafout is essential for predicting level of damage from a false spring event. We will refer to the timing of these collective phenophases (i.e. budburst to leafout) as the duration of vegetative risk. The duration of vegetative risk is usually extended if a freezing event occurs during the phenophases between budburst and full leafout \citep{Augspurger2009}, which could result in exposure to multiple frost events in one season.

Climate change further complicates understanding species vulnerabilities to false spring events. Most species are expected to begin leafout earlier in the season with earlier warming spring temperatures but some species may have the opposite response due to less winter chilling or decreased photoperiod cues \citep{Cleland2006, Yu2010, Xin2016}. Generally, individuals that initiate budburst earlier in the spring may attempt to limit freezing risk by decreasing the duration of vegetative risk in order to minimize the exposure of less frost tolerant phenophases. But with a changing climate and thus shifts in phenological cues, this relationship may change. Additionally, various studies that investigate latitudinal effects indicate that species growing further north respond to a different interaction of cues than those growing further south and, subsequently, species across different regions may have different durations of vegetative risk \citep {Partanen2004, Viheraaarnio2006, Caffarra2011}. Studies also suggest that species within the same system can exhibit different sensitivies to the three cues \citep{Basler2012, Laube2013} thus further amplifying the myriad of climatic and phenological shifts as well as the varying species-level effects.  We assessed climate data across North America and Europe, long-term observational data, and experimental data to gain a better understanding of the the interaction between duration of vegetative risk and false spring events in an attempt to unravel these complexities.

\subsection {Predictable Regional Differences in Climate, Species Responses and False Spring Risk}
Numerous studies have investigated the the relationship between budburst and the interaction of cues by using latitudinal gradients \citep{Partanen2004, Viheraaarnio2006, Caffarra2011, Zohner2016, Gauzere2017}, however few have integrated longitudinal variation or regional effects. Yet climate and thus false spring risk varies across regions. For example, consider five archetypal regions within a temperate climate. Some regions may experience harsher winters and greater temperature variability throughout the year, and these more variable regions often have a much higher risk of false spring (i.e. Maine) than others (i.e. Lyon) (Figure \ref{fig:regional}). Understanding and integrating such spatiotemporal effects and regional differences when investigating false spring risk and duration of vegetative risk would help improve predictions as climate change progresses.

Accurate predictions need to carefully consider how chilling and forcing, which are key drivers of budburst and leafout, vary significantly across a longitudinal gradient. Some studies indicate that populations further inland will initiate budburst first, whereas those closer to the coast will initiate budburst later in the season and that the distance from the coast is a stronger indicator of budburst timing than latitude \citep{Myking2007}. Climatic variation across regions and at different distances from the coast results in varying durations of vegetative risk due to different chilling and forcing temperatures. It is therefore important to recognize climate regime extremes (e.g. seasonal trends, annual minima and annual maxima) across regions in future studies in order to better understand the interplay between duration of vegetative risk and climatic variation. The climatic implications of advancing forcing temperatures could potentially lead to earlier dates of budburst and enhance the risk of frost. These shifts in climatic regimes could vary in intensity across regions (i.e. habitats currently at risk of false spring damage could become low risk regions over time). 

There are also discrepancies in defining a false spring event related to understanding the temperature threshold for damage. Some regions and species may tolerate lower temperature thresholds than others (Figure \ref{fig:temp}). Not only is there debate on what a damaging temperature is, but it is still not well understood how the damage sustained relates to the duration of the frost \citep{Sakai1987, Augspurger2009, Vitasse2014, Vitra2017}. It is crucial to gain an understanding on which climatic parameters result in false spring events and how these parameters may vary across regions. It is anticipated that most regions will have earlier spring onsets, however, last freeze dates will not advance at the same rate \citep{Inouye2008,Martin2010,Labe2016}, rendering some regions and species to be more susceptible to false spring events in the future. 

\subsection{Changes in Phenological Cues and the Duration of Vegetative Risk}
The risk of false spring may shift as climate change progresses and greater forcing temperatures occur earlier in the year. Temperate trees utilize two phases of dormancy: endodormancy, when trees are inhibited from growing, and ecodormancy, when trees can grow if the external environment is conducive \citep{Basler2012}. However, it is unclear when precisely the plants enter the ecodormancy phase \citep{Chuine2016}. With spring advancing, trees will potentially oscillate between chilling and forcing cues \citep{Martin2010} and, thus,  extend the number of required growing degree days necessary for budburst to occur. Studies also suggest that spring forcing temperatures directly affect the duration of vegetative risk: years with lower forcing temperatures and fewer growing degree days will have longer durations of vegetative risk \citep{Donnelly2017}. Therefore, it is hypothesized that the species able to track the shifts in spring advancement due to climate change will be more susceptible to false spring damage \citep{Scheifinger2003}. We investigated this interaction using observational data from Harvard Forest \citep{Okeefe2014} and compared two years of data: one year that was thermally late (1997) and another year that was thermally early (2012).

By comparing the two years, we found that the durations of vegetative risk contrasted, with most species in 2012 having longer durations than those in 1997. In 2012, a false spring event was reported across many regions of the US and at Harvard Forest low freezing temperatures were recorded on the 29th of April, after many species had initiated budburst (Figure \ref{fig:hf}). This contrast across years could be due to the less consistent forcing temperatures after budburst in 2012, lower photoperiod cues, the false spring event or it could be a combination of the three depending on the species. The effects of spring forcing temperatures on the duration of vegetative risk varies across species, which could indicate variation in physiological cues that drive budburst and influence the duration of vegetative risk.

Each species responds differently to climate change, therefore, the duration of vegetative risk depends on the interaction between cues and species. Species dominated by forcing cues may shift earlier and earlier with climate change but most species also have photoperiod and chilling cues, which complicate predictions. For example, as climate change progresses, higher spring forcing temperatures may be required for species experiencing insufficient winter chilling (due to warmer winter temperatures), especially at lower latitudes \citep{McCreary1990, Morin2009, Fu2012, Polgar2014, Chuine2010}. Anthropogenic climate change will cause changes in winter and spring temperatures, resulting in greater differences in spring phenology cue requirements across species and regions. This interaction of cues---and how climate change will affect that interaction---is crucial for recognizing which species will likely become more at risk of false spring events in the future.

We assessed data from a growth chamber experiment in order to investigate the interaction of cues across species and predict potential shifts in duration of vegetative risk with climate change. We compared 11 temperate forest species between two treatments: high chilling hours, long photoperiod and high forcing temperatures (strong treatment effects) against no additional chilling, short photoperiod and low forcing temperatures (weak treatment effects) (Flynn and Wolkovich, 2017?). According to the results, all individuals have longer durations of vegetative risk with the weaker treatment effects (Figure \ref{fig:dan}A.) Our results indicate forcing temperatures and photoperiod cues have bigger effects on the duration of vegetative risk than over-winter chilling and with more forcing or longer daylengths, the rate of leafout is expected to shorten (Figure \ref{fig:dan}B.). This could suggest that chilling influences budburst and leafout similarly, while photoperiod and forcing temperatures have varying effects on the two phenophases. With a changing climate, forcing temperatures will increase and initiate earlier in the season while photoperiod cues will remain stagnant or decrease. This cue interaction could potentially elongate the duration of vegetative risk and expose at risk plants to more intense false spring events or even multiple events in one year. Further studies are essential to investigate the interplay between chilling, forcing, and photoperiod cues on the duration of vegetative risk, especially for species occupying ecological niches more susceptible to false spring events. 

\section{Conclusion}
Temperate forest trees are most at risk to frost damage in the spring due to the stochasticity of spring freezes. With warm temperatures advancing in the spring but last spring freeze dates advancing at a slower rate, there could potentially be more damaging false spring events in the future, especially in high risk regions \citep{Gu2008, Inouye2008}. The current equation for evaluating false spring damage (Equation \ref{eq:1}) largely simplifies the myriad of complexities involved in assessing false spring damage and risks inadequately predicting future trends. More studies are necessary to gain an understanding of relationships between species avoidance and tolerance strategies, climatic regimes, and physiological cue interactions with the duration of vegetative risk. It is also essential that a temperature threshold is established across functional types and phenophases in order to effectively predict false spring risk in the future. An integrated approach to assessing past and future spring freeze damage would offer more robust predictions as climate change progresses, which is essential in order to mitigate the adverse ecological and economic effects of false springs.

\nocite{Soudani2012}
\nocite{White2009}
\nocite{Schaber2005}
\nocite{Schwartz1993}
\nocite{Barker2005}
\nocite{Sanchez2013}
\nocite{Longstroth2012}
\nocite{Barlow2015}
\nocite{Longstroth2013}
\bibliography{..//refs/SpringFreeze.bib}


\begin{figure}[H]

{\centering \includegraphics[width=\maxwidth]{figure/fsifig-1} 

}

\caption[A scatterplot indicating FSI values from 2008 to 2014 for each methdology used in this study]{A scatterplot indicating FSI values from 2008 to 2014 for each methdology used in this study. USA-NPN FSI values are green (USA-NPN, 2016), observed FSI values are blue (O'Keefe, 2014), and PhenoCam FSI values are red (Richardson, 2015).}\label{fig:fsifig}
\end{figure}



\begin{figure}[H]

{\centering \includegraphics[width=\maxwidth]{figure/risk-1} 

}

\caption{A figure showing the differences in spring phenology and false spring risk across two species: \textit{Ilex mucronata} (L.) and \textit{Betula alleghaniensis} (Marsh.). We mapped a possible false spring event based on historic weather data and compared it to the observational data collected at Harvard Forest (O'Keefe, 2014). In this scenario, the \textit{Ilex mucronata} would be exposed to a false spring event, whereas the \textit{Betula alleghaniensis} would avoid it entirely. DVR stands for Duration of Vegetative Risk.}\label{fig:risk}
\end{figure}



\begin{figure} [H] 
 -\begin{center}
 -\includegraphics[width=16cm, height=13cm]{..//figure/RegRisk_clean.pdf} 
 -\caption{A comparison of false spring risk across five climate regions. By determining the average time of budburst to leafout dates for the dominant species in five archetypal climate regions, we were able to estimate the current spatial variation of false spring risk. We assessed the number of freeze days (-2.2$^{\circ}$C) (Schwartz, 1993) that occurred on average over the past 50 years within the average durations of vegetative risk for each region (USA-NPN, 2016; Soudani \textit{et al.}, 2012; White \textit{et al.}, 2009; Schaber \& Badeck, 2005).}\label{fig:regional} 
 -\end{center}
 -\end{figure}

\begin{figure}[H]

{\centering \includegraphics[width=\maxwidth]{figure/temp-1} 

}

\caption[A comparison of damaging spring freezing temperature thresholds across ecological and agronomic studies]{A comparison of damaging spring freezing temperature thresholds across ecological and agronomic studies. Each study is listed on the y axis along with the taxonimic group of focus. Next to the species name is the freezing definition used within that study (e.g. 100\% is 100\% lethality). Each point is the best estimate recorded for the temperature threshold with standard deviation if indicated in the study. The shape of the point represents the phenophases of interest and the colors indicate the type of study (i.e. agronomic or ecological).}\label{fig:temp}
\end{figure}



\begin{figure}[H]

{\centering \includegraphics[width=\maxwidth]{figure/hf-1} 

}

\caption[Duration of vegetative risk for 8 species at Harvard Forest, comparing 1997 and 2012]{Duration of vegetative risk for 8 species at Harvard Forest, comparing 1997 and 2012. In 1997, the aggregated GDDs to budburst were the lowest and the durations of vegetative risk overall were shorter, whereas in 2012, the aggregated GDDs to budburst were the highest and the durations of vegetative risk were longer. The dotted line indicates a false spring event in 2012. The histogram at the bottom right corner indicates the frequency of accumulated GDDs to budburst for each year and indicates that 1997 was a thermally late year and 2012 was a thermally early year. }\label{fig:hf}
\end{figure}



\begin{figure} [H] 
 -\begin{center}
 -\includegraphics[width=16cm, height=13cm]{..//figure/DVRdiff_twoplots.pdf} 
 -\caption{Results from the growth chamber experiment. (A) Is a comparison of the durations of vegetative risk across two treatments for each species collected for the experiment. Species along the x-axis are ordered by day of budburst. Data was collected from a growth chamber experiment where one treatment had no additional overwinter chilling, low spring forcing temperatures, and shorter spring daylengths and the other had additional overwinter chilling, high spring forcing temperatures, and longer spring daylenghts. The standard error is represented by the bars around each point. (B) A plot of the parameter effects on the duration of vegetative risk. Spring forcing temperatures had the largest effect on the rate of leafout, with photoperiod also being a critical factor. However, they were offset by the interactions, especially the forcing and photoperiod interaction.}\label{fig:dan} 
 -\end{center}
 -\end{figure}
 

\end{document}
