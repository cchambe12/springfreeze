\documentclass{article}\usepackage[]{graphicx}\usepackage[]{color}
%% maxwidth is the original width if it is less than linewidth
%% otherwise use linewidth (to make sure the graphics do not exceed the margin)
\makeatletter
\def\maxwidth{ %
  \ifdim\Gin@nat@width>\linewidth
    \linewidth
  \else
    \Gin@nat@width
  \fi
}
\makeatother

\definecolor{fgcolor}{rgb}{0.345, 0.345, 0.345}
\newcommand{\hlnum}[1]{\textcolor[rgb]{0.686,0.059,0.569}{#1}}%
\newcommand{\hlstr}[1]{\textcolor[rgb]{0.192,0.494,0.8}{#1}}%
\newcommand{\hlcom}[1]{\textcolor[rgb]{0.678,0.584,0.686}{\textit{#1}}}%
\newcommand{\hlopt}[1]{\textcolor[rgb]{0,0,0}{#1}}%
\newcommand{\hlstd}[1]{\textcolor[rgb]{0.345,0.345,0.345}{#1}}%
\newcommand{\hlkwa}[1]{\textcolor[rgb]{0.161,0.373,0.58}{\textbf{#1}}}%
\newcommand{\hlkwb}[1]{\textcolor[rgb]{0.69,0.353,0.396}{#1}}%
\newcommand{\hlkwc}[1]{\textcolor[rgb]{0.333,0.667,0.333}{#1}}%
\newcommand{\hlkwd}[1]{\textcolor[rgb]{0.737,0.353,0.396}{\textbf{#1}}}%
\let\hlipl\hlkwb

\usepackage{framed}
\makeatletter
\newenvironment{kframe}{%
 \def\at@end@of@kframe{}%
 \ifinner\ifhmode%
  \def\at@end@of@kframe{\end{minipage}}%
  \begin{minipage}{\columnwidth}%
 \fi\fi%
 \def\FrameCommand##1{\hskip\@totalleftmargin \hskip-\fboxsep
 \colorbox{shadecolor}{##1}\hskip-\fboxsep
     % There is no \\@totalrightmargin, so:
     \hskip-\linewidth \hskip-\@totalleftmargin \hskip\columnwidth}%
 \MakeFramed {\advance\hsize-\width
   \@totalleftmargin\z@ \linewidth\hsize
   \@setminipage}}%
 {\par\unskip\endMakeFramed%
 \at@end@of@kframe}
\makeatother

\definecolor{shadecolor}{rgb}{.97, .97, .97}
\definecolor{messagecolor}{rgb}{0, 0, 0}
\definecolor{warningcolor}{rgb}{1, 0, 1}
\definecolor{errorcolor}{rgb}{1, 0, 0}
\newenvironment{knitrout}{}{} % an empty environment to be redefined in TeX

\usepackage{alltt}
\usepackage{Sweave}
\usepackage{float}
\usepackage{graphicx}
\usepackage{tabularx}
\usepackage{siunitx}
\usepackage{geometry}
\usepackage{pdflscape}
\usepackage{mdframed}
\usepackage{natbib}
\usepackage{bibentry}
\bibliographystyle{..//refs/styles/besjournals.bst}
\usepackage[small]{caption}
\setlength{\captionmargin}{30pt}
\setlength{\abovecaptionskip}{0pt}
\setlength{\belowcaptionskip}{10pt}
\topmargin -1.5cm        
\oddsidemargin -0.04cm   
\evensidemargin -0.04cm
\textwidth 16.59cm
\textheight 21.94cm 
%\pagestyle{empty} %comment if want page numbers
\parskip 7.2pt
\renewcommand{\baselinestretch}{1.5}
\parindent 0pt
\usepackage{lineno}
\linenumbers

\newmdenv[
  topline=true,
  bottomline=true,
  skipabove=\topsep,
  skipbelow=\topsep
]{siderules}
\IfFileExists{upquote.sty}{\usepackage{upquote}}{}
\begin{document}
\nobibliography*
\noindent \textbf{\Large{Rethinking False Spring Risk: Submission Questions}}\\
\vspace{3ex}

\noindent \textbf{What is the scientific question you are addressing?} \\

%\noindent Climate change has brought renewed interest to late spring freeze events, commonly called false springs, which shape species ranges and life history strategies. Combining theory from ecology, climatology, physiology, biogeography and crop science we ask: what are the effects of false springs and what are the major factors that drive plants' risk to frost damage? Our questions are framed with the goal of advancing forecasting. \\ % See if my changes are okay. 65 Words
\noindent Climate change has renewed interest in late spring freezes, commonly called false springs, which shape species ranges and life history strategies. Combining theory from various fields in an attempt to advance forecasting we ask: what are the effects of false springs and what factors drive plants' risk to frost damage? \\ % Now 50 words


\noindent \textbf{What is/are the key finding(s) that answers this question?} \\

\noindent We find that current definitions for false springs remain generally simple. Most definitions fail to incorporate factors such as species, functional group, phenological cue requirements, regional effects and other climatic regimes. We argue that a new approach that integrates these and other crucial factors is needed. \\ % EMW: Be sure to give citation correctly in online formatting! (I only say this as I have screwed it up before). 46 words

\noindent \textbf{Why is this work important and timely?}\\

%\noindent There has been a growing number of studies that take a simplified view of false springs, which can lead to incorrect forecasting. This manuscript is especially timely because new methods are essential to predict the effects of false spring events across diverse species and climate regimes, especially under climate change. \\ %If you have space you should start much stronger here. I would start with something like:  .... (or something exciting like that). % 50 Words

\noindent Recent studies have documented how climate change is reshaping late spring freezes, with cascading ecological and economic impacts. Most studies, however, take a simplified view of false springs, which can lead to incorrect forecasting. New methods are essential to predict the effects of false spring events, especially under climate change. \\

\noindent \textbf{ Does your paper fall within the scope of GCB; what biological AND global change aspects does it address?}\\

\noindent The manuscript will demonstrate how an integrated view of false spring that incorporates the complexity of factors underlying plant strategies to frost would rapidly advance progress in this field, including improved predictions of spring freeze risk with global change, and, novel insights into how plants are shaped by spring frost. \\ % 50 words

\noindent \textbf{What are the three most recently published papers that are relevant to this question?} \\

\bibentry{Vitra2017}\\
\bibentry{Lenz2016}\\
\bibentry{Liu2018}\\ % We want three papers in journals with equally high or higher impact factors than GCB. So I think these are good choices!

\noindent \textbf{ If you listed non-preferred reviewers, please provide a justification for each.} \\

\noindent N/A \\

\noindent \textbf{ If your manuscript does not conform to author or formatting guidelines (e.g. exceeding word limit), please provide a justification.} \\

\noindent N/A \\


\bibliography{..//refs/SpringFreeze.bib}

\end{document}
