\documentclass[11pt,a4paper]{article}\usepackage[]{graphicx}\usepackage[]{color}
%% maxwidth is the original width if it is less than linewidth
%% otherwise use linewidth (to make sure the graphics do not exceed the margin)
\makeatletter
\def\maxwidth{ %
  \ifdim\Gin@nat@width>\linewidth
    \linewidth
  \else
    \Gin@nat@width
  \fi
}
\makeatother

\definecolor{fgcolor}{rgb}{0.345, 0.345, 0.345}
\newcommand{\hlnum}[1]{\textcolor[rgb]{0.686,0.059,0.569}{#1}}%
\newcommand{\hlstr}[1]{\textcolor[rgb]{0.192,0.494,0.8}{#1}}%
\newcommand{\hlcom}[1]{\textcolor[rgb]{0.678,0.584,0.686}{\textit{#1}}}%
\newcommand{\hlopt}[1]{\textcolor[rgb]{0,0,0}{#1}}%
\newcommand{\hlstd}[1]{\textcolor[rgb]{0.345,0.345,0.345}{#1}}%
\newcommand{\hlkwa}[1]{\textcolor[rgb]{0.161,0.373,0.58}{\textbf{#1}}}%
\newcommand{\hlkwb}[1]{\textcolor[rgb]{0.69,0.353,0.396}{#1}}%
\newcommand{\hlkwc}[1]{\textcolor[rgb]{0.333,0.667,0.333}{#1}}%
\newcommand{\hlkwd}[1]{\textcolor[rgb]{0.737,0.353,0.396}{\textbf{#1}}}%
\let\hlipl\hlkwb

\usepackage{framed}
\makeatletter
\newenvironment{kframe}{%
 \def\at@end@of@kframe{}%
 \ifinner\ifhmode%
  \def\at@end@of@kframe{\end{minipage}}%
  \begin{minipage}{\columnwidth}%
 \fi\fi%
 \def\FrameCommand##1{\hskip\@totalleftmargin \hskip-\fboxsep
 \colorbox{shadecolor}{##1}\hskip-\fboxsep
     % There is no \\@totalrightmargin, so:
     \hskip-\linewidth \hskip-\@totalleftmargin \hskip\columnwidth}%
 \MakeFramed {\advance\hsize-\width
   \@totalleftmargin\z@ \linewidth\hsize
   \@setminipage}}%
 {\par\unskip\endMakeFramed%
 \at@end@of@kframe}
\makeatother

\definecolor{shadecolor}{rgb}{.97, .97, .97}
\definecolor{messagecolor}{rgb}{0, 0, 0}
\definecolor{warningcolor}{rgb}{1, 0, 1}
\definecolor{errorcolor}{rgb}{1, 0, 0}
\newenvironment{knitrout}{}{} % an empty environment to be redefined in TeX

\usepackage{alltt}
\usepackage[top=1.00in, bottom=1.0in, left=1.1in, right=1.1in]{geometry}
\usepackage{graphicx}
\usepackage[numbers]{natbib}
\bibliographystyle{..//refs/styles/nature.bst}
\usepackage[export]{adjustbox}

% 1. Co-author names and affiliations.
%2. A point-by-point summary (~300-600 words) outlining what will be discussed in the article and why it is timely.
%3. A list of 10-20 key recent references (published in the past 2-4 years) that indicate the intended breadth and balance of the proposed article.
\IfFileExists{upquote.sty}{\usepackage{upquote}}{}
\begin{document}

\noindent Dear Dr. Craze:
\vspace{1.5ex}\\
\noindent Please consider our manuscript titled `Rethinking False Spring' as an Opinion piece for \textit{Trends in Ecology and Evolution}. We combine theory from ecology, climatology, physiology, biogeography and crop science to examine the effects of late spring freezing events -- or false springs -- and the complexity of factors that drive plants' risk to damage. The aim of the manuscript is to help advance forecasting in climate change and ecological studies. \\

\noindent Temperate tree and shrub species are at risk of damage from late spring freezing events, also known as false springs. However the extent of damage and the frequency and intensity of these events is still largely unknown. Due to shifts in climate, biological spring onset is advancing and many temperate tree and shrub species are initiating leafout 4-6 days earlier per $^{\circ}$C of warming \citep{Wolkovich2012, Polgar2014} but last spring freeze dates are not predicted to advance at the same rate as spring onset in some regions \citep{Labe2016}. Climate change could potentially amplify the effects of false springs in these regions, which could result in highly adverse ecological and economic consequences \citep{Ault2013, Vitra2017}. \\

\noindent Many studies have reported false spring events in recent years and have linked these events to climate change \citep[e.g.][]{Augspurger2013, Menzel2015}. Recent false spring events have led to a growing body of research investigating the effects on temperate forests and agricultural crops. However, current definitions for false springs are generally simple -- i.e. budburst occurs before the last spring freeze \citep{Gu2008}. This simple definition assumes consistency of damage across species, functional group, life stages, and other climatic regimes, ignoring that such factors can greatly impact plants' false spring risk. \\

\noindent This manuscript is especially timely because new methods are essential to properly evaluate the effects of false spring events across the diverse species and climate regimes, especially under climate change. The ultimate intent of this manuscript is to demonstrate how an integrated view of false spring that incorporates the complexity factors would rapidly advance progress in this field, improve predictions of spring freeze risk under a changing climate, and, potentially, provide novel insights to how plants respond to and are shaped by spring frost. \\

\noindent Our author team provides an international and interdisciplinary approach. Because our manuscript cuts across the fields of ecology, crop science, biogeography and climatology our authorship list is slightly longer than allowed -- at four authors -- we found this was necessary to bring a robust perspective from each field. We hope that you will find it suitable for publication in \textit{Trends in Ecology and Evolution}. \\

\noindent Please find a list of key references below. This Opinion piece is not under consideration for publication elsewhere. Thank you for your consideration. \\

\noindent Sincerely, \\
\vspace{1.5ex}\\
\noindent Catherine Chamberlain

\newpage
\nocite{Vitasse2014}
\nocite{Vitasse2014a}
\nocite{Zohner2016}
\nocite{Lenz2016}
\nocite{Allstadt2015}
\nocite{Hofmann2015}
\nocite{Kollas2014}
\nocite{Peterson2014}
\nocite{Xin2016}
\nocite{Lenz2013}
\nocite{Muffler2016}
\bibliography{..//refs/SpringFreeze.bib}

\end{document}
