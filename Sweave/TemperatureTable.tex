\documentclass{article}\usepackage[]{graphicx}\usepackage[]{color}
%% maxwidth is the original width if it is less than linewidth
%% otherwise use linewidth (to make sure the graphics do not exceed the margin)
\makeatletter
\def\maxwidth{ %
  \ifdim\Gin@nat@width>\linewidth
    \linewidth
  \else
    \Gin@nat@width
  \fi
}
\makeatother

\definecolor{fgcolor}{rgb}{0.345, 0.345, 0.345}
\newcommand{\hlnum}[1]{\textcolor[rgb]{0.686,0.059,0.569}{#1}}%
\newcommand{\hlstr}[1]{\textcolor[rgb]{0.192,0.494,0.8}{#1}}%
\newcommand{\hlcom}[1]{\textcolor[rgb]{0.678,0.584,0.686}{\textit{#1}}}%
\newcommand{\hlopt}[1]{\textcolor[rgb]{0,0,0}{#1}}%
\newcommand{\hlstd}[1]{\textcolor[rgb]{0.345,0.345,0.345}{#1}}%
\newcommand{\hlkwa}[1]{\textcolor[rgb]{0.161,0.373,0.58}{\textbf{#1}}}%
\newcommand{\hlkwb}[1]{\textcolor[rgb]{0.69,0.353,0.396}{#1}}%
\newcommand{\hlkwc}[1]{\textcolor[rgb]{0.333,0.667,0.333}{#1}}%
\newcommand{\hlkwd}[1]{\textcolor[rgb]{0.737,0.353,0.396}{\textbf{#1}}}%
\let\hlipl\hlkwb

\usepackage{framed}
\makeatletter
\newenvironment{kframe}{%
 \def\at@end@of@kframe{}%
 \ifinner\ifhmode%
  \def\at@end@of@kframe{\end{minipage}}%
  \begin{minipage}{\columnwidth}%
 \fi\fi%
 \def\FrameCommand##1{\hskip\@totalleftmargin \hskip-\fboxsep
 \colorbox{shadecolor}{##1}\hskip-\fboxsep
     % There is no \\@totalrightmargin, so:
     \hskip-\linewidth \hskip-\@totalleftmargin \hskip\columnwidth}%
 \MakeFramed {\advance\hsize-\width
   \@totalleftmargin\z@ \linewidth\hsize
   \@setminipage}}%
 {\par\unskip\endMakeFramed%
 \at@end@of@kframe}
\makeatother

\definecolor{shadecolor}{rgb}{.97, .97, .97}
\definecolor{messagecolor}{rgb}{0, 0, 0}
\definecolor{warningcolor}{rgb}{1, 0, 1}
\definecolor{errorcolor}{rgb}{1, 0, 0}
\newenvironment{knitrout}{}{} % an empty environment to be redefined in TeX

\usepackage{alltt}
%\usepackage{Sweave}
\usepackage{float}
\usepackage{graphicx}
\usepackage{tabularx}
\usepackage{siunitx}
\usepackage{mdframed}
\usepackage{natbib}
\bibliographystyle{..//refs/styles/besjournals.bst}
\usepackage[small]{caption}
\setkeys{Gin}{width=0.8\textwidth}
\setlength{\captionmargin}{30pt}
\setlength{\abovecaptionskip}{0pt}
\setlength{\belowcaptionskip}{10pt}
\topmargin -1.5cm        
\oddsidemargin -0.04cm   
\evensidemargin -0.04cm
\textwidth 16.59cm
\textheight 21.94cm 
%\pagestyle{empty} %comment if want page numbers
\parskip 7.2pt
\renewcommand{\baselinestretch}{1.5}
\parindent 0pt

\newmdenv[
  topline=true,
  bottomline=true,
  skipabove=\topsep,
  skipbelow=\topsep
]{siderules}

%% R Script


\IfFileExists{upquote.sty}{\usepackage{upquote}}{}
\begin{document}
\title{Rethinking False Spring Risk}
\author{Chamberlain, Wolkovich}
\date{\today}
\maketitle 

\renewcommand{\thetable}{\arabic{table}}
\renewcommand{\thefigure}{\arabic{figure}}
\renewcommand{\labelitemi}{$-$}
%%%%%%%%%%%%%%%%%%%%%%%%%%%%%%%%%%%%%%%%%%%%%%%%%%%%%%%%%%%%%%%%
\section*{Temperature Thresholds for Damage: Agricultural vs Ecological}
\begin{center}
\captionof{table}{Comparing damaging spring temperature thresholds in ecological and agronomical studies across various species and phenophases.} \label{tab:title} 
\begin{tabular}{c c c c c c c}
\hline
\textbf{Sector} & \textbf{Phenophase} & \textbf{Species} & \textbf{Temperature ($^{\circ}$C)} & \textbf{Type} & \textbf{Notes} & \textbf{Source} \\
\hline
Ecological & Budburst to Leafout & Sorbus aucuparia & -7.4 & 50\% lethality & experiment designed to measure specific temperature threshold by species & %\citep{Lenz2013, Lenz2016} \\
Ecological & Budburst to Leafout & Prunus avium & -8.5 & 50\% lethality & experiment designed to measure specific temperature threshold by species & %\citep{Lenz2013, Lenz2016}\\
Ecological & Budburst to Leafout & Tilia platyphyllos & -7.4 & 50\% lethality & experiment designed to measure specific temperature threshold by species & %\citep{Lenz2013, Lenz2016}\\
Ecological & Budburst to Leafout & Acer pseudoplatanus & -6.7 & 50\% lethality & experiment designed to measure specific temperature threshold by species & %\citep{Lenz2013, Lenz2016}\\
Ecological & Budburst to Leafout & Fagus sylvatica & -4.8 & 50\% lethality & experiment designed to measure specific temperature threshold by species & %\citep{Lenz2013, Lenz2016}\\
Ecological & Spring Onset & All plant functional types & -2.2 & hard & Date of spring onset - Last hard freeze date = damage index & %\citep{Schwartz93\\
Ecological & Spring Onset & All plant functional types & -1.7 & soft & varying degress of damage across edges and interiors of forests & %\citep{Augspurger2013} \\
Ecological & All phenophases & All plant functional types & 2 SD below winter TAVG & cold-air outbreaks & 2 or more days at 2+ standard deviations below averge winter temperature & %\cite{Vavrus2006} \\
Ecological & After budburst & Eucalyptus pauciflora & -5.8 & climate change & elevated CO2 study indicates higher levels of CO2 at same temperature cause greater damage to leaves & %\cite{Barker2005} \\
Ecological & After budburst & All plant functional types & -2.2 & damaging & use 7 day threshold between last freeze day after date of budburst as significant & %\cite{Peterson2014} \\
Agrinomical & After budburst & All plant functional types & 2 & Risk threshold & tissue temperature can be 1-3 degrees lower on clear, still nights & %\cite{Cannell1986} \\
Agrinomical & Flowers & Vaccinium spp. & -4.4 to 0 & sprinkler protection threshold & tissue temperature can be 1-3 degrees lower on clear, still nights & Longstroth, 2013 \\
Agrinomical & Budburst & Rosaceae & -7.2 & 10\% lethality & specifically apple and pear trees & Longstroth, 2012\\
Agrinomical & Budburst & Rosaceae & -13.3 & 90\% lethality & specifically apple and pear trees & Longstroth, 2012 \\
Agrinomical & All phenophases & All plant functional types & Radiation Frost & Frost formation when temperature matches dew point & Occurs on calm, clear nights and boudary-layer inversion occurs & Barlow et al., 2015; Andresen, 2009 \\
Agrinomical & Flowers & Wheat & -4 to -5 & 10-90\% lethality & if nighttime temperature drops by 1℃, reaches 90\% lethality & Barlow et al., 2015 \\
Agrinomical & Vegetative & Wheat & -7 for 2hrs & 100\% lethality & From -4$^{\circ}$C to -7$^{\circ}$C during vegetative growth, 0-100\% lethality & Barlow et al., 2015 \\
Agrinomical & Vegetative & Rice & 4.7 & lethal limit & minimum temperature for 100\% lethality & Sanchez et al., 2014 \\
Agrinomical & Vegetative & Corn & -1.8 & lethal limit & minimum temperature for 100\% lethality & Sanchez et al., 2014\\
Agrinomical & Vegetative & Wheat & -17.2 & lethal limit & minimum temperature for 100\% lethality & Sanchez et al., 2014 \\
\hline
\end{tabular}
\end{center}


\bibliography{..//refs/SpringFreeze.bib}
\end{document}
