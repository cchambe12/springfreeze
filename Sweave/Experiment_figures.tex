\documentclass{article}\usepackage[]{graphicx}\usepackage[]{color}
%% maxwidth is the original width if it is less than linewidth
%% otherwise use linewidth (to make sure the graphics do not exceed the margin)
\makeatletter
\def\maxwidth{ %
  \ifdim\Gin@nat@width>\linewidth
    \linewidth
  \else
    \Gin@nat@width
  \fi
}
\makeatother

\definecolor{fgcolor}{rgb}{0.345, 0.345, 0.345}
\newcommand{\hlnum}[1]{\textcolor[rgb]{0.686,0.059,0.569}{#1}}%
\newcommand{\hlstr}[1]{\textcolor[rgb]{0.192,0.494,0.8}{#1}}%
\newcommand{\hlcom}[1]{\textcolor[rgb]{0.678,0.584,0.686}{\textit{#1}}}%
\newcommand{\hlopt}[1]{\textcolor[rgb]{0,0,0}{#1}}%
\newcommand{\hlstd}[1]{\textcolor[rgb]{0.345,0.345,0.345}{#1}}%
\newcommand{\hlkwa}[1]{\textcolor[rgb]{0.161,0.373,0.58}{\textbf{#1}}}%
\newcommand{\hlkwb}[1]{\textcolor[rgb]{0.69,0.353,0.396}{#1}}%
\newcommand{\hlkwc}[1]{\textcolor[rgb]{0.333,0.667,0.333}{#1}}%
\newcommand{\hlkwd}[1]{\textcolor[rgb]{0.737,0.353,0.396}{\textbf{#1}}}%
\let\hlipl\hlkwb

\usepackage{framed}
\makeatletter
\newenvironment{kframe}{%
 \def\at@end@of@kframe{}%
 \ifinner\ifhmode%
  \def\at@end@of@kframe{\end{minipage}}%
  \begin{minipage}{\columnwidth}%
 \fi\fi%
 \def\FrameCommand##1{\hskip\@totalleftmargin \hskip-\fboxsep
 \colorbox{shadecolor}{##1}\hskip-\fboxsep
     % There is no \\@totalrightmargin, so:
     \hskip-\linewidth \hskip-\@totalleftmargin \hskip\columnwidth}%
 \MakeFramed {\advance\hsize-\width
   \@totalleftmargin\z@ \linewidth\hsize
   \@setminipage}}%
 {\par\unskip\endMakeFramed%
 \at@end@of@kframe}
\makeatother

\definecolor{shadecolor}{rgb}{.97, .97, .97}
\definecolor{messagecolor}{rgb}{0, 0, 0}
\definecolor{warningcolor}{rgb}{1, 0, 1}
\definecolor{errorcolor}{rgb}{1, 0, 0}
\newenvironment{knitrout}{}{} % an empty environment to be redefined in TeX

\usepackage{alltt}
\usepackage{Sweave}
\usepackage{float}
\usepackage{graphicx}
\usepackage{tabularx}
\usepackage{siunitx}
\usepackage{mdframed}
\usepackage{amsmath}
\usepackage{gensymb}
\usepackage{natbib}
\bibliographystyle{..//refs/styles/besjournals.bst}
\usepackage[small]{caption}
\setkeys{Gin}{width=0.8\textwidth}
\setlength{\captionmargin}{30pt}
\setlength{\abovecaptionskip}{0pt}
\setlength{\belowcaptionskip}{10pt}
\topmargin -1.5cm        
\oddsidemargin -0.04cm   
\evensidemargin -0.04cm
\textwidth 16.59cm
\textheight 21.94cm 
%\pagestyle{empty} %comment if want page numbers
\parskip 7.2pt
\renewcommand{\baselinestretch}{1.5}
\parindent 0pt

\newmdenv[
  topline=true,
  bottomline=true,
  skipabove=\topsep,
  skipbelow=\topsep
]{siderules}

%% R Script


\IfFileExists{upquote.sty}{\usepackage{upquote}}{}
\begin{document}

\renewcommand{\thetable}{\arabic{table}}
\renewcommand{\thefigure}{\arabic{figure}}
\renewcommand{\labelitemi}{$-$}
%%%%%%%%%%%%%%%%%%%%%%%%%%%%%%%%%%%%%%%%%%%%%%%%%%%%%%%%%%%%%%%%%%%%%%%%%%%%%%%%%%%%%%%%%%%
\section*{US-NPN Timeline Figures}

\begin{figure} [H]
\begin{center}
\caption{Day of budburst and the day of leaf out for native tree species in New England. Data was collected from a growth chamber experiment using any combination of two photoperiod treatments, two forcing treatments, and three chilling treatments. The standard deviation is represented in blue for budburst and green for leaf out. }
\includegraphics{..//output/Dan_TXandSp.pdf} 
\end{center}
\end{figure}


Anova Table (Type II tests)

Response: risk
                     Sum Sq Df F value Pr(>F)
as.factor(chilling) 308.113  2  2.9281 0.1440
force                24.894  1  0.4732 0.5221
photoperiod           9.000  1  0.1711 0.6963
Residuals           263.067  5               
Anova Table (Type II tests)

Response: risk
                    Sum Sq Df F value   Pr(>F)   
as.factor(chilling)  24.24  2  0.1965 0.823653   
force               586.06  1  9.5038 0.007578 **
photoperiod         361.86  1  5.8681 0.028544 * 
Residuals           924.99 15                    
---
Signif. codes:  0 '***' 0.001 '**' 0.01 '*' 0.05 '.' 0.1 ' ' 1
\begin{kframe}

{\ttfamily\noindent\bfseries\color{errorcolor}{\#\# Error in Anova.lm(lm(risk \textasciitilde{} as.factor(chilling) + force + photoperiod, : residual df = 0}}\end{kframe}% latex table generated in R 3.3.1 by xtable 1.8-2 package
% Tue Apr 11 13:10:57 2017
\begin{table}[ht]
\centering
\begin{tabular}{lrrrr}
  \hline
 & Sum Sq & Df & F value & Pr($>$F) \\ 
  \hline
as.factor(chilling) & 24.24 & 2 & 0.20 & 0.8237 \\ 
  force & 586.06 & 1 & 9.50 & 0.0076 \\ 
  photoperiod & 361.86 & 1 & 5.87 & 0.0285 \\ 
  Residuals & 924.99 & 15 &  &  \\ 
   \hline
\end{tabular}
\end{table}



\end{document}
