\documentclass{article}\usepackage[]{graphicx}\usepackage[]{color}
%% maxwidth is the original width if it is less than linewidth
%% otherwise use linewidth (to make sure the graphics do not exceed the margin)
\makeatletter
\def\maxwidth{ %
  \ifdim\Gin@nat@width>\linewidth
    \linewidth
  \else
    \Gin@nat@width
  \fi
}
\makeatother

\definecolor{fgcolor}{rgb}{0.345, 0.345, 0.345}
\newcommand{\hlnum}[1]{\textcolor[rgb]{0.686,0.059,0.569}{#1}}%
\newcommand{\hlstr}[1]{\textcolor[rgb]{0.192,0.494,0.8}{#1}}%
\newcommand{\hlcom}[1]{\textcolor[rgb]{0.678,0.584,0.686}{\textit{#1}}}%
\newcommand{\hlopt}[1]{\textcolor[rgb]{0,0,0}{#1}}%
\newcommand{\hlstd}[1]{\textcolor[rgb]{0.345,0.345,0.345}{#1}}%
\newcommand{\hlkwa}[1]{\textcolor[rgb]{0.161,0.373,0.58}{\textbf{#1}}}%
\newcommand{\hlkwb}[1]{\textcolor[rgb]{0.69,0.353,0.396}{#1}}%
\newcommand{\hlkwc}[1]{\textcolor[rgb]{0.333,0.667,0.333}{#1}}%
\newcommand{\hlkwd}[1]{\textcolor[rgb]{0.737,0.353,0.396}{\textbf{#1}}}%
\let\hlipl\hlkwb

\usepackage{framed}
\makeatletter
\newenvironment{kframe}{%
 \def\at@end@of@kframe{}%
 \ifinner\ifhmode%
  \def\at@end@of@kframe{\end{minipage}}%
  \begin{minipage}{\columnwidth}%
 \fi\fi%
 \def\FrameCommand##1{\hskip\@totalleftmargin \hskip-\fboxsep
 \colorbox{shadecolor}{##1}\hskip-\fboxsep
     % There is no \\@totalrightmargin, so:
     \hskip-\linewidth \hskip-\@totalleftmargin \hskip\columnwidth}%
 \MakeFramed {\advance\hsize-\width
   \@totalleftmargin\z@ \linewidth\hsize
   \@setminipage}}%
 {\par\unskip\endMakeFramed%
 \at@end@of@kframe}
\makeatother

\definecolor{shadecolor}{rgb}{.97, .97, .97}
\definecolor{messagecolor}{rgb}{0, 0, 0}
\definecolor{warningcolor}{rgb}{1, 0, 1}
\definecolor{errorcolor}{rgb}{1, 0, 0}
\newenvironment{knitrout}{}{} % an empty environment to be redefined in TeX

\usepackage{alltt}
\usepackage{Sweave}
\usepackage{float}
\usepackage{graphicx}
\usepackage{tabularx}
\usepackage{siunitx}
\usepackage{geometry}
\usepackage{pdflscape}
\usepackage{mdframed}
\usepackage{natbib}
\bibliographystyle{..//refs/styles/besjournals.bst}
\usepackage[small]{caption}
\setlength{\captionmargin}{30pt}
\setlength{\abovecaptionskip}{0pt}
\setlength{\belowcaptionskip}{10pt}
\topmargin -1.5cm        
\oddsidemargin -0.04cm   
\evensidemargin -0.04cm
\textwidth 16.59cm
\textheight 21.94cm 
%\pagestyle{empty} %comment if want page numbers
\parskip 7.2pt
\renewcommand{\baselinestretch}{1.5}
\parindent 0pt
\usepackage{lineno}
\linenumbers

\newmdenv[
  topline=true,
  bottomline=true,
  skipabove=\topsep,
  skipbelow=\topsep
]{siderules}

%% R Script
\begin{kframe}


{\ttfamily\noindent\bfseries\color{errorcolor}{\#\# Error in library(stargazer): there is no package called 'stargazer'}}\end{kframe}

\IfFileExists{upquote.sty}{\usepackage{upquote}}{}
\begin{document}

\renewcommand{\thetable}{\arabic{table}}
\renewcommand{\thefigure}{\arabic{figure}}
\renewcommand{\labelitemi}{$-$}
\setkeys{Gin}{width=0.8\textwidth}

%%%%%%%%%%%%%%%%%%%%%%%%%%%%%%%%%%%%%%%%%%%%%%%
\begin{table}
\begin{siderules}
{\large{\textbf {Box 1: } }}
\newline
Cold hardiness (i.e. freezing tolerance) is essential for all temperate plants in order to survive cold winters and stochastic freezes \citep{Vitasse2014}.
\begin{description}
\item \textbf{Cold Hardiness:} Ability to resist injury to low temperatures
\item \textbf{Cold Acclimation: } Adjustment period of freezing tolerance by decreasing risk of intracellular freezing through various mechanisms \citep{Charrier2011}
\item \textbf{Cold Deacclimation: } Dehardening of buds and increase in metabolism and development \citep{Vitasse2014}
\end{description}

 \begin{flushright}
 \includegraphics[width=15cm, height=9cm]{..//figure/Hardiness_DVR.pdf}
 \end{flushright}
\begin{description} 
\item \textbf{Sept-Nov (Orange): } During the cold acclimation phase, cold hardiness in the bud increases rapidly as temperate plants begin to enter dormancy.
\item \textbf{Nov-Feb (Blue): } Once buds reach the dormancy phase, buds are able to tolerate temperates as low as -25$^{\circ}$C to -40$^{\circ}$C or lower \citep{Charrier2011,Vitasse2014}.
\item \textbf{Feb-May (Green): } Freezing tolerance diminishes again during the cold deacclimation phase once buds begin to swell (-8$^{\circ}$C) and is lowest between budburst (-2$^{\circ}$C) to leafout (-3$^{\circ}$C).
\end{description}
\end{siderules}
\end{table}

\bibliography{..//refs/SpringFreeze.bib}
\end{document}
